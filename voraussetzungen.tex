\subsection{Grundlagen}
\label{sec:voraussetzungen}
\subsectionframe

\begin{frame}{Kompilieren}
	\todo{Kompilieren erklären? Nötig?}\\
	\todo{FLOSS Prozess erklären?}\\
\end{frame}

\begin{frame}{Freie Software}
	\todo{Farben, kanten abrunden}
	\begin{center}
		\begin{tikzpicture}
		[
			node distance = 2 cm,
			box/.style = {thick, draw=black, fill=red, align=center, inner sep = 0.3cm},
		]
		
		\visible<1-> {
			\node[box] (freedom) {Die Freiheit zu};
		}
		
		\visible<2-> {
			\node[box, above left of = freedom] {Verwenden\only<handout>{\\Die Freiheit, die Software uneingeschränkt und für jeden Zweck einzusetzen.}};
		}
	
		\visible<3-> {
			\node[box, above right of = freedom] {Verstehen\only<handout>{\\Die Freiheit, die Funktionsweise der Software untersuchen und verstehen zu können.}};
		}
	
		\visible<4-> {
			\node[box, below left of = freedom] {Verbreiten\only<handout>{\\Die Freiheit, Kopien der Software zu verbreiten, um damit seinen Mitmenschen zu helfen.}};
		}
	
		\visible<5-> {
			\node[box, below right of = freedom] {Verbessern\only<handout>{\\Die Freiheit, die Software zu verbessern und die Verbesserungen an die Öffentlichkeit weiterzugeben, sodass die gesamte Gesellschaft davon profitieren kann.}};
		}
	
		\end{tikzpicture}
	\end{center}
\end{frame}
\note
{
	\url{http://www.inf-schule.de/software/freie\_software/02\_freiheiten\_freier\_software}
}

\begin{frame}{FLOSS}
	\todo{Farben}\\
	\begin{center}
		\begin{tikzpicture}
		[
			node distance = 1.8 cm,
			box/.style = {thick, draw=black, fill=red, align=center, inner sep = 0.25cm},
			background/.style = {draw=blue},
		]
		\small

		\node (center) {};

		\visible<8->{
			\node[box, above = 2.25cm of center] (floss) {FLOSS: Free/Libre Open Source Software};
		}

		\visible<2->{
			\node[box, left = 1.25cm of center] (free) {Freie Software};
		}
		\visible<3->{
			\node[box, above left of = free] (user) {Anwender};
			\draw (free) -- (user);
			\node[box, above right of = free] (community) {Gesellschaft};
			\draw (free) -- (community);
		}
		\visible<4->{
			\node[box, below left of = free] (ethical) {ethisch};
			\draw (free) -- (ethical);
			\node[box, below right of = free] (social) {sozial};
			\draw (free) -- (social);
		}

		\visible<5->{
			\node[box, right = 1.25cm of center] (open) {Open Sources};
		}
		\visible<6->{
			\node[box, above left of = open] (developer) {Entwickler};
			\draw (open) -- (developer);
			\node[box, above right of = open] (business) {Business};
			\draw (open) -- (business);
		}
		\visible<7->{
			\node[box, below left of = open] (practical) {praktisch};
			\draw (open) -- (practical);
			\node[box, below right of = open] (pragmatic) {pragmatisch};
			\draw (open) -- (pragmatic);
		}

		\begin{pgfonlayer}{background} 
			\node[box] [fit = (floss) (free) (open) (user) (community) (ethical) (social) (developer) (business) (practical) (pragmatic)] {};
		\end{pgfonlayer}
		
		\end{tikzpicture}
	\end{center}
\end{frame}
\note
{
	\url{https://de.wikipedia.org/wiki/Freie\_Software\#Open\_Source}
}

\begin{frame}{Proprietär versus Kommerziell}
	\todo{entfernen?}
	\begin{center}
		\begin{tabular}{c|c|c|}
			 & \thead{proprietär} &  \thead{frei} \\ 
			\hline 
			\thead{kommerziell} & \makecell{Windows\\Mac OS} & \makecell{Red Hat Enterprise\\Linux (RHEL)}\\ 
			\hline 
			\thead{gratis} & \makecell{(iOS)\\Freeware} & \makecell{Debian} \\ 
			\hline 
		\end{tabular} 
	\end{center}
\end{frame}

\begin{frame}{IP}
	\begin{center}
		Geistiges Eigentum
	\end{center}
\end{frame}
\note
{
	(IP, Intellectual Property, Geistiges Eigentum) \url{https://de.wikipedia.org/wiki/Geistiges\_Eigentum\#Systematik}
}


\begin{frame}{IP - Patent}
	\begin{itemize}
		\item \todo{positives Beispiel}
		\item \todo{Beispiel mp3?}
		\item \todo{Private versus Geschäftlich?}
		\item \todo{Softwarepatente?}
	\end{itemize}
\end{frame}
\note
{
	(IP, Intellectual Property, Geistiges Eigentum) \url{https://de.wikipedia.org/wiki/Geistiges\_Eigentum\#Systematik}
	\begin{itemize}
		\item Patent: \url{https://de.wikipedia.org/wiki/Patent}
		\item Defensivpublikation
	\end{itemize}
}

\begin{frame}{IP - Markenrecht}
	\todo{GNU Logo}
\end{frame}
\note
{
	\begin{itemize}
		\item Markenrecht: \url{https://de.wikipedia.org/wiki/Marke_(Recht)}
		\item Firefox und Debian (Iceweasel): \url{https://lwn.net/Articles/676799/}
		\item GNU Logo ist geschützt: \url{https://www.gnu.org/graphics/agnuhead.html}
	\end{itemize}
}

\begin{frame}{IP - Urheberrecht}
	\todo{Verweis auf Bild vom Anfang}\\
	\todo{Software}
\end{frame}
\note
{
	\begin{itemize}
		\item Urheberrecht: \url{https://de.wikipedia.org/wiki/Urheberrecht}
		\item gilt automatisch: \url{https://anwalt-im-netz.de/urheberrecht/urheberrecht-faq.html\#wann}
		\item Copyrightzeichen ist nicht nötig: \url{https://de.wikipedia.org/wiki/Copyrightzeichen}
		\item Nicht verzichtbar (kein public domain, Gemeinfreiheit: \url{https://de.wikipedia.org/wiki/Gemeinfreiheit\#Entlassung\_in\_die\_Gemeinfreiheit})
	\end{itemize}
}


\begin{frame}{Copyleft}
	\todo{viral erklären}\\
	\todo{Zeichen}
\end{frame}
\note
{
	\begin{itemize}
		\item Copyleft, viral: https://de.wikipedia.org/wiki/Copyleft
	\end{itemize}
}


