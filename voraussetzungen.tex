\subsection{Einführung}
\subsectionframe

\begin{frame}{Kompilieren}
	\todo{Kompilieren erklären? Nötig?}
\end{frame}

\begin{frame}{Freie Software}
	\todo{Farben, kanten abrunden}
	\begin{center}
		\begin{tikzpicture}
		[
			node distance = 2 cm,
			box/.style = {thick, draw=black, fill=red, align=center, inner sep = 0.3cm},
		]
		
		\visible<1-> {
			\node[box] (freedom) {Die Freiheit zu};
		}
		
		\visible<2-> {
			\node[box, above left of = freedom] {Verwenden\only<handout>{\\Die Freiheit, die Software uneingeschränkt und für jeden Zweck einzusetzen.}};
		}
	
		\visible<3-> {
			\node[box, above right of = freedom] {Verstehen\only<handout>{\\Die Freiheit, die Funktionsweise der Software untersuchen und verstehen zu können.}};
		}
	
		\visible<4-> {
			\node[box, below left of = freedom] {Verbreiten\only<handout>{\\Die Freiheit, Kopien der Software zu verbreiten, um damit seinen Mitmenschen zu helfen.}};
		}
	
		\visible<5-> {
			\node[box, below right of = freedom] {Verbessern\only<handout>{\\Die Freiheit, die Software zu verbessern und die Verbesserungen an die Öffentlichkeit weiterzugeben, sodass die gesamte Gesellschaft davon profitieren kann.}};
		}
	
		\end{tikzpicture}
	\end{center}
	\url{http://www.inf-schule.de/software/freie\_software/02\_freiheiten\_freier\_software}
\end{frame}

\begin{frame}{FLOSS}
	\todo{Farben}\\
	\todo{Linien zu Eigenschaften}\\
	\todo{Node Gruppen um Freie Software resp. Open Source}\\
	\begin{center}
		\begin{tikzpicture}
		[
			node distance = 2 cm,
			box/.style = {thick, draw=black, fill=red, align=center, inner sep = 0.3cm},
		]
		
		\node[box] (floss) {FLOSS: Free/Libre Open Source Software};

		\node[box, below left of = floss] (free) {Freie Software};
		\node[box, above left of = free] {Anwender};
		\node[box, above right of = free] {Gesellschaft};
		\node[box, below left of = free] {ethisch};
		\node[box, below right of = free] {sozial};
	
		\node[box, below right of = floss] (open) {Open Sources};
		\node[box, above left of = open] {Entwickler};
		\node[box, above right of = open] {Business};
		\node[box, below left of = open] {praktisch};
		\node[box, below right of = open] {pragmatisch};
		
		\begin{pgfonlayer}{background} 
			\node[box] [fit = (floss) (free) (open)] {};
		\end{pgfonlayer}
		
		\end{tikzpicture}
	\end{center}
	\url{https://de.wikipedia.org/wiki/Freie\_Software\#Open\_Source}
\end{frame}

\begin{frame}{Proprietär versus Kommerziell}
	\todo{entfernen?}
	\begin{center}
		\begin{tabular}{|c|c|c|}
			\hline 
			 & proprietär &  frei \\ 
			\hline 
			kommerziell & Windows, Mac OS & Red Hat Enterprise Linux (RHEL) \\ 
			\hline 
			gratis & Freeware, (iOS) & Debian \\ 
			\hline 
		\end{tabular} 
	\end{center}
\end{frame}

\begin{frame}{IP - Geistiges Eigentum}
	\begin{itemize}
		\item Markenrecht
		\begin{itemize}
			\item \todo{Tux oder GNU}
		\end{itemize}
		\item Patent
		\begin{itemize}
			\item \todo{positives Beispiel}
			\item \todo{Beispiel mp3?}
			\item \todo{Private versus Geschäftlich?}
			\item \todo{Softwarepatente?}
		\end{itemize}
		\item Urheberrecht
		\begin{itemize}
			\item \todo{Verweis auf Bild vom Anfang}
		\end{itemize}
	\end{itemize}
\end{frame}
\note
{
	(IP, Intellectual Property, Geistiges Eigentum) \url{https://de.wikipedia.org/wiki/Geistiges\_Eigentum\#Systematik}
	\begin{itemize}
		\item Markenrecht
		\begin{itemize}
			\item Markenrecht: \url{https://de.wikipedia.org/wiki/Marke_(Recht)}
			\item Firefox und Debian (Iceweasel): \url{https://lwn.net/Articles/676799/}
		\end{itemize}
		\item Patent
		\begin{itemize}
			\item Patent: \url{https://de.wikipedia.org/wiki/Patent}
			\item Defensivpublikation
		\end{itemize}
		\item Urheberrecht
		\begin{itemize}
			\item Urheberrecht: \url{https://de.wikipedia.org/wiki/Urheberrecht}
			\item gilt automatisch: \url{https://anwalt-im-netz.de/urheberrecht/urheberrecht-faq.html\#wann}
			\item Copyrightzeichen ist nicht nötig: \url{https://de.wikipedia.org/wiki/Copyrightzeichen}
			\item Nicht verzichtbar (kein public domain, Gemeinfreiheit: \url{https://de.wikipedia.org/wiki/Gemeinfreiheit\#Entlassung\_in\_die\_Gemeinfreiheit})
		\end{itemize}
	\end{itemize}
}

\begin{frame}{Copyleft}
	\begin{itemize}
		\item viral
	\end{itemize}
\end{frame}
\note
{
	\begin{itemize}
		\item Copyleft, viral: https://de.wikipedia.org/wiki/Copyleft
	\end{itemize}
}


